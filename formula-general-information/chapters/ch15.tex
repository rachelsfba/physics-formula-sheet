\begin{longtable}{p{0.5\textwidth} p{0.5\textwidth}}
  \tablesection{Chapter 15: Electric Forces \& Electric Fields}
  \tablesubsection{Coulomb's Law}
	\(\pm e\) & \textit{The Elementary Charge}; an object can have a charge of $\pm 1e$, $\pm 2e$ and so on, or a fractional charge $\pm0.5e$, $\pm0.22e$ and so on. The charge of the electron $e = \SI{-1.60219e-19}{\coulomb}$ Coulombs. The charge of the proton is $+\SI{1.60e-19}{\coulomb}$. The charge of the neutron is \SI{0}{\coulomb}. The coulomb is the SI unit of charge \\
  	\(F=k_e\displaystyle\frac{\abs{q_1}\abs{q_2}}{r^2}\) & Yields the magnitude of the electric force $F$ between charges $q_1$ and $q_2$ separated by distance $r$ where $k_e=\SI{8.9875e9}{\newton\meter\squared\per\coulomb\squared}$ is Coulomb's Constant \\

  \tablesubsection{Electric Fields}
  
  \(\vec{E}\equiv\displaystyle\frac{\vec{F}}{q_0}\) & Yields the magnitude of the electric field $\vec{E}$ produced by a charge $Q$ at the location of a small ``test'' charge $q_0$ where $\vec{F}$ is the force exerted by $Q$ on $q_0$, measured in \si{\newton\per\coulomb} \\
  \(E=k_e\displaystyle\frac{\abs{q}}{r^2}\) & An alternate form of the equation yielding the magnitude of the electric field as related to Coulomb's law \\
  \(q=ne\) & Yields the magnitude of a charge $q$ where $n$ is the number of charged particles and $e=\SI{1.60e-19}{\coulomb}$ is the elementary charge \\

  \notabene{An electric field exists at a point if an arbitrarily small test charge at that point is subject to an electric force there. If equal test charges are placed at $x=a$ and $x=-a$, the electric field is $0$ at the origin, by symmetry.}

  \notabene{A \textbf{conductor in electrostatic equilibrium} has the following properties:%enumerate is inside notabene
	\begin{enumerate}
		\item The electric field is zero everywhere inside the conducting material.
		\item Any excess charge on an isolated conductor must reside entirely on its surface.
		\item The electric field just outside a charged conductor is perpendicular to the conductor's surface.
		\item On an irregularly shaped conductor, charge accumulates where the radius of curvature of the surface is smallest, at sharp points.\vspace{1cm}
	\end{enumerate}}
    
    
    
  \tablesubsection{Electric Flux \& Gauss's Law}

    \(\Phi_E=EA\) & Yields the electric flux $\Phi$ of electric field $E$ passing through an area $A$ perpendicular to the field. Electric flux is a rearranged form of $E\propto\frac{N}{A}$ where $N$ is the number of electric field lines passing through area $A$. Electric flux is measured in \si{\newton\meter\squared\per\coulomb} \\
    \(\Phi_E=EA\cos\theta\) & Yields electric flux when the surface in consideration is at an angle $\theta$ with respect to the field \\
    \(\Phi_E=4\pi k_eq=\displaystyle\frac{q}{\epsilon_0}\) & Yields electric flux through a closed spherical surface surrounding a charge $q$ where $\epsilon_0=\frac{1}{4\pi k_e}=\SI{8.85e-12}{\coulomb\squared\per\newton\per\meter\squared}$ is the permittivity of free space \\
    \(k_e=\frac{1}{4\pi\epsilon_0}\) & The relationship between Coulomb's constant $k_e$ and the permittivity of free space $\epsilon_0$ \\
    \(E \propto \frac{N}{A}\) & The magnitude of the electric field $E$ is proportional to the number of electric lines $N$ per unit area $A$. This can be rewritten to $N \propto EA$ \\
    \(EA = \Phi_E = \frac{Q_{inside}}{\epsilon_0}\) & \textbf{Gauss's Law}'s states that the electric flux through any closed surface is equal to the net charge $Q$ inside the surface divided by the permittivity of free space, $\eta_0$. For highly symmetric distributions of charge, Gauss's Law can be used to calculate electric fields \\
\end{longtable}
%%% Local Variables:
%%% mode: latex
%%% TeX-master: "main"
%%% End: