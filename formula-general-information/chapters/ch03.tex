\begin{longtable}{p{0.5\textwidth} p{0.5\textwidth}}
	\tablesection{Chapter 3: Vectors and Two-Dimensional Motion}  
    
	\tablesubsection{Resultant Vector Formul\ae}
    
	\(v_{res} = \displaystyle\sqrt{\left(\sum v_x\right)^2 + \left(\sum v_y\right)^2}\) & An application of the Pythagorean theorem which yields the magnitude of the resultant velocity $v_{res}$ between two or more velocity vectors broken into $x$- and $y$-components \\
	\(\theta_{res} = \arctan\left(\displaystyle\frac{\sum v_y}{\sum v_x}\right)\) & The resultant angle $\theta_{res}$ between two or more velocity vectors, broken into $x$- and $y$-components \\
	\(\theta_{opp} = \arctan\left(\displaystyle\frac{\sum v_y}{\sum v_x}\right) + \SI{180}{\degree}\) & The angle opposite the resultant velocity vector $\theta_{opp}$ \\
	
	\notabene{These same formul\ae\space applied to velocity $\vec{v}$ can be applied to displacement $\Delta\vec{x}$ and to acceleration $\vec{a}$}
	
	\tablesubsection{Displacement, Velocity, and Acceleration in Two Dimensions}
	
	\(\Delta\vec{r}\equiv\vec{r}_f - \vec{r}_i\) & The displacement $\Delta\vec{r}$ is the change in the position vector of an object \\
	\(\vec{v}_{avg}\equiv\displaystyle\frac{\Delta\vec{r}}{\Delta t}\) & The average velocity $\vec{v}_{avg}$ with displacement $\Delta\vec{r}$ across time interval $\Delta t$ in \si{\meter\per\second} \\
	\(\vec{v}\equiv\displaystyle\lim_{\Delta t\to 0}\frac{\Delta\vec{r}}{\Delta t}\) & The instantaneous velocity $\vec{v}$ \\ \\%white space between formulas
	\(\vec{a}_{avg}\equiv\displaystyle\frac{\Delta\vec{v}}{\Delta t}\) & The average acceleration $\vec{a}_{avg}$ with change in velocity $\Delta\vec{f}$ across time interval $\Delta t$ in \si{\meter\per\second\squared} \\
	\(\vec{a}\equiv\displaystyle\lim_{\Delta t\to 0}\frac{\Delta\vec{v}}{\Delta t}\) & The instantaneous acceleration $\vec{a}$ \\ \\%adds white space between formulas
        \(R\equiv\Delta x = \displaystyle\frac{2v_i\sin\theta_i}{g}\) & The \textit{range} equation; yields the maximum horizontal displacement of a projectile where $y_i = y_f$ and the only acceleration acting on the object is $g$ \\
	
	\tablesubsection{Vector Applications of Polar Conversion Formul\ae}
	\vspace{2mm}
	\begin{tabular}{c c c c}
		\(\Delta x = d\cos\theta\) & \(v_x = v\cos\theta\) & \(a_x = a\cos\theta\) & \(F_x = F\cos\theta\) \\
		\(\Delta y = d\sin\theta\) & \(v_y = v\sin\theta\) & \(a_y = a\sin\theta\) & \(F_y = F\sin\theta\) \\
	\end{tabular} & \\ \\%white space to force the \tablesubsection{} to fall entirely on the next page
	
	\tablesubsection{Component Vector Formul\ae}
	
	\begin{tabular}{c c}
		\(v_{xf} = v_{xi} + a_xt\) & \(\Delta x = v_{xi}t + \frac{1}{2}a_xt^2\) \\
		\(v_{yf} = v_{yi} + a_yt\) & \(\Delta y = v_{yi}t + \frac{1}{2}a_yt^2\) \\
	\end{tabular} & \\

\end{longtable}
%%% Local Variables:
%%% mode: latex
%%% TeX-master: "main"
%%% End: