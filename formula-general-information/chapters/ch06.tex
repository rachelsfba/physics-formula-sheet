\begin{longtable}{p{0.5\textwidth} p{0.5\textwidth}}
  \tablesection{Chapter 6: Momentum \& Collisions}
  \tablesubsection{Momentum \& Impulse}
  
     \( \vec{p} \equiv m\vec{v} \) & The linear momentum \(\vec{p}\) of an object of mass $m$ moving with velocity \(\vec{v}\) is the product of its mass and velocity.  This is measured in \si{\kilo\gram\meter\per\second} \\
   \(KE = \displaystyle\frac{p^2}{2m}\) & implicitly, we arrive at \(p = \displaystyle\sqrt{\left(KE\right)\left(2m\right)}\) \\
   \(\displaystyle\sum\vec{p}_{system} = \vec{p}_1 + \vec{p}_2 + \ldots + \vec{p}_n\) & The sum of linear momentum of a system can be expressed as the algebraic sum of all individual linear momenta of that system \\
   \(\displaystyle\vec{F}_{net} = m\vec{a} = m\frac{\Delta\vec{v}}{\Delta t} = \frac{\Delta\left(m\vec{v}\right)}{\Delta t} = \frac{\Delta\vec{p}}{\Delta t}\) & Newton's second law and momentum; the change in an object's momentum \(\Delta \vec{p}\) divided by the time interval \(\Delta t\) yields the constant net force \(\vec{F}_{net}\) acting upon the object \\
   \(\vec{J} = \vec{F}\Delta t = \Delta\vec{p} = m\Delta\vec{v} = m\vec{v}_f - m\vec{v}_i\) & Impulse-momentum theorem; Thus impulse $\vec{J}$ is the change in momentum measured in \si{\newton\second} \\
   \(\vec{F}_{avg} \Delta t = \Delta\vec{p}\) & Alternate form of the impulse-momentum theorem \\
  
    \tablesubsection{Conservation of Momentum}

  \(m_1\vec{v}_{1i} + m_2\vec{v}_{2i} = m_1\vec{v}_{1f} + m_2\vec{v}_{2f}\) & The law of conservation of momentum for two objects interacting in a system. This can be expanded to any number of objects interacting in a system \\
  \notabene{When no net external force acts on a system, the total momentum of the system remains constant in time}

  \tablesubsection{Collisions}

  \(v_f = \displaystyle\frac{m_1v_{1i} + m_2v_{2i}}{m_1 + m_2}\) & Yields the final velocity for two objects in a perfectly inelastic collision \\
  \(m_1v_{1i} + m_2v_{2i} = m_1v_{1f} + m_2v_{2f}\) & \\
  \(\frac{1}{2}m_1v_{1i}^2 + \frac{1}{2}m_2v_{2i}^2 = \frac{1}{2}m_1v_{1f}^2 + \frac{1}{2}m_2v_{2f}^2\)& The two conditions required for elastic collisions, since their momenta (first condition) and kinetic energy (second condition) are both conserved. \\
	\(v_{1i} - v_{2i} = -\left(v_{1f} - v_{2f}\right)\) & The relationship between velocities in a perfectly elastic head-on collision \\
	\(v_{1f} = \displaystyle (\frac{m_1 - m_2}{m_1 + m_2})v_{1i}\) & Applies to head-on elastic collisions \\ \\%spacing between equations
	\(v_{2f} = \displaystyle (\frac{2m_1}{m_1 + m_2})v_{1i}\) & Applies to head-on elastic collisions \\

  \notabene{\textit{Inelastic Collisions} are collisions in which momentum is conserved, but kinetic energy is not. In a \textit{Perfectly Inelastic Collision}, two objects collide but remain attached after the collision so their final velocities are the same. \textit{Elastic Collisions} are collisions in which both momentum and kinetic energy are conserved. For example, two objects collide and bounce off of one another after the collision}

  \tablesubsection{Glancing Collisions}

  \(m_1v_{1ix} + m_2v_{2ix} = m_1v_{1fx} + m_1v_{2fx}\) & \\
  \(m_1v_{1iy} + m_2v_{2iy} = m_1v_{1fy} + m_1v_{2fy}\) & Component formul\ae\space for glancing collisions between two objects \\
  
  \notabene{In glancing collisions problems, object 1 moves at an angle $\theta$ with respect to the horizontal while object 2 moves at an angle $\phi$ with respect to the horizontal}
  
  \tablesubsection{Center of Mass}
  
  \(X_{cm}=\displaystyle\frac{\sum m_x}{\sum m}=\frac{m_1x_1+m_2x_2+\ldots+m_nx_n}{m_1+m_2+\ldots+m_n}\) & Yields the $x$-coordinate of the center of mass \\ \\%additional spacing between lines
  \(Y_{cm}=\displaystyle\frac{\sum m_y}{\sum m}=\frac{m_1y_1+m_2y_2+\ldots+m_ny_n}{m_1+m_2+\ldots+m_n}\) & Yields the $y$-coordinate of the center of mass \\

  \tablesubsection{Rocket Propulsion}

  \(\Delta v = v_e\displaystyle\ln\left(\frac{M_i}{M_f}\right)\) & Tsiolkovsky rocket equation; yields the potential change in velocity $\Delta v$ where $M_i$ is the mass of the rocket plus the initial fuel mass, $M_f$ is the mass of the rocket plus the remaining fuel mass and $v_e$ is the velocity of the exhaust relative to the rocket \\
  \(Ma = M\displaystyle\frac{\Delta v}{\Delta t} = \abs{v_e\frac{\Delta M}{\Delta t}}\) & Yields instantaneous thrust where $\Delta M$ is the change in rocket mass due to fuel loss, $v_e$ is the velocity of the exhaust relative to the rocket, and $\Delta t$ is the time interval \\
\end{longtable}
%%% Local Variables:
%%% mode: latex
%%% TeX-master: "main"
%%% End: