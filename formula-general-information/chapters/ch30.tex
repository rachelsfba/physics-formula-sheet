\begin{longtable}{p{0.5\textwidth} p{0.5\textwidth}}
  \tablesection{Chapter 30: Nuclear Energy \& Elementary Particles}
  \tablesubsection{Nuclear Fission}

  \(\ce{^1_0n + ^{235}_{92}U -> ^{236}_{92}U\textrm{*} -> X + Y + neutrons}\) & An example of nuclear fission with \ce{^{235}U} \\

  \notabene{Nuclear fission occurs when a heavy nucleus, such as \ce{^{235}U} splits, or fissions, into two smaller nuclei. In such a reaction, the total mass of the products is less than the original mass of the heavy nucleus}

  \tablesubsection{Nuclear Fusion}

  \(\ce{^1_1H + ^1_1H -> ^2_1D + e+ + \nu}\) &\\
  \(\ce{^1_1H + ^2_1D -> ^3_2He + \gamma}\) & The steps in the proton-proton fusion cycle where $D$ stands for deuterium \(\left(\ce{^2_1H}\right)\) \\

  \notabene{The first step in a nuclear fusion process is proton-proton fusion (the fusion of two Hydrogen atoms)}

  \(\ce{^1_1H + ^3_2He -> ^4_2He + e+ + \nu}\) & The hydrogen-helium fusion reaction \\
  \(\ce{^3_2He + ^3_2He -> ^4_2He + 2}\left(\ce{^1_1H}\right)\) & The helium-helium fusion reaction \\

  \notabene{The second step in a nuclear fusion process is wither hydrogen-helium fusion or helium-helium fusion}
  \notabene{Nuclear fusion occurs when two light nuclei combine to form a heavier nucleus. Unlike nuclear fission, nuclear fusion is an energy source not yet harnessed by humans}

  \tablesubsection{Classification of Particles}

  \notabene{\textit{Hadrons} are particles which interact through the strong force. There are two primary classes of hadrons, \textit{mesons} and \textit{baryons} distinguished by their masses and spins. Today, it is believed that hadrons are composed of quarks}
  \notabene{Mesons are known to decay finally into electrons, positrons, neutrinos, and photons. A good example of a meson is the pion ($\pi$), the lightest of the known mesons with a mass of approximately \SI{140}{\mega\electronvolt\per\lightspeed\squared} and a spin of 0. The decay of pions is as follows: \[\ce{\pi- -> \mu- + \bar{\nu}}\]\[\ce{\mu- -> e- + \nu\textrm{} + \bar{\nu}}\]}%the textrm is there because \ce wants \nu + to be \nu+ for some reason
  \notabene{Baryons have masses equal to or greater than the proton mass (``baryon'' means ``heavy'' in Greek) and their spin is always a non-integer value $\left(\frac{1}{2}\textrm{ or }\frac{3}{2}\right)$. Protons and neutrons are baryons, as are many other particles. With the exception of the proton, all baryons decay in such a way that the end products include a proton. For example the hyperon $\Xi$ decays first into a \ce{\Lambda^0} then a \ce{\pi-}}
  \notabene{\textit{Leptons} (from the Greek leptos, meaning ``light'') are a group of particles that participate in the weak interaction. All leptons have a spin of $\frac{1}{2}$. Included in this group are electrons, muons, and neutrinos, which are all less massive than the lightest hadron. Although hadrons have size and structure, leptons appear to be truly elementary, with no structure down to the limit of resolution of experiment (about $10^{-19}$\,\si{\meter}). There are currently only six known leptons, the electron, the muon, the tau, and a neutrino associated with each:}
  \multicolumn{2}{c}{
    \begin{tabular}{c c c}
      \(\displaystyle\bfrac{e^-}{\nu_e}\) & \(\displaystyle\bfrac{\mu^-}{\nu_\mu}\) & \(\displaystyle\bfrac{\tau^-}{\nu_\tau}\)
    \end{tabular}
    }
\end{longtable}
%%% Local Variables:
%%% mode: latex
%%% TeX-master: "main"
%%% End: