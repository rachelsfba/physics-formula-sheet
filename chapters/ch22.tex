\begin{longtable}{p{0.5\textwidth} p{0.5\textwidth}}
  \tablesection{Chapter 22: Reflection \& Refraction of Light}
  \tablesubsection{The Nature of Light}

  \(E=hf\) & Yields the energy of a photon $E$ with frequency $f$ where $h=6.63\e{-34}$\,\si{\joule} is Planck's constant \\

  \tablesubsection{Reflection \& Refraction}

  \(\theta_1^\prime=\theta_1\) & Relates the angle of reflection $\theta_1$ to the angle of incidence $\theta_1^\prime$ where the two angles form a \SI{90}{\degree} angle with one another \\

  \notabene{When light travels from one medium to another, its frequency does not change}

  \(n\equiv\displaystyle\frac{c}{v}=\frac{\lambda_0}{\lambda_n}\) & Yields the index of refraction $n$ of a medium where $v$ is the speed of light in that medium and $\lambda_0$ is the wavelength of light in a vacuum and $\lambda_n$ is the wavelength of light in the medium \\

  \notabene{The index of refraction of a vacuum $n=1$}

  \(\lambda_1n_1=\lambda_2n_2\) & Relates the wavelengths of an electromagnetic wave in two different media with indexes of refraction $n_1$ and $n_2$ \\
  \(n_1\sin\theta_1=n_2\sin\theta_2\) & Snell's Law of Refraction \\
  \(\sin\theta_c=\displaystyle\frac{n_2}{n_1}\) for \(n_1>n_2\) & Yields the critical angle $\theta_c$, the angle of incidence at which the refraced light ray moves parallel to the boundary so that $\theta_2=$\SI{90}{\degree}. For angles of incidence greater than $\theta_c$, the light bean is entirely reflected at the boundary \\

  \notabene{Total internal reflection occurs only when light is incident on the boundary of a medium having a lower index of refraction than the medium in which it is traveling. If $n_1<n_2$, then $\sin\theta_c>1$ which is impossible, because the greatest possible value for the sine of an angle is 1}
\end{longtable}
%%% Local Variables:
%%% mode: latex
%%% TeX-master: "../main"
%%% End: