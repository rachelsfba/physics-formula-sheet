\begin{longtable}{p{0.5\textwidth} p{0.5\textwidth}}
  \tablesection{Chapter 13: Vibrations \& Waves}
  \tablesubsection{Simple Harmonic Motion}

  \(\vec{F}_s=-k\Delta\vec{x}\) & Hooke's Law where $k$ is the spring constant and $\Delta\vec{x}$ is the displacement of the spring from equilibrium. Because Hooke's Law is a restoring force (in that it always pushes or pulls the object toward the equilibrium position) it can be used to describe simple harmonic motion \\
  \(\vec{a} = -\displaystyle\frac{k}{m}\Delta\vec{x} = -\vec{\omega}^2\Delta\vec{x}\) & Yields the acceleration of an object moving with simple harmonic motion \\
  \(\displaystyle\vec{a}_{max}=\frac{k}{m}A=\vec{\omega}^2A\) & Yields the maximum acceleration for an object in SHM where $A$ is the amplitude|the maximum distance of the object from its equilibrium position where $\Delta\vec{x}=\pm A$ \\
  \(\displaystyle\vec{v} = \pm\sqrt{\frac{k}{m}\left(A^2-\Delta\vec{x}^2\right)} = \pm\omega\sqrt{A^2-\Delta\vec{x}^2}\) & Yields the velocity of an object moving with simple harmonic motion \\
  \(\displaystyle\vec{v}_{max} = \pm\sqrt{\frac{k}{m}A^2} = \pm\vec{\omega}\sqrt{A^2}\) & Yields the maximum velocity for an object in SHM where $\Delta\vec{x}=0$ \\

  \tablesubsection{Elastic Potential Energy}

  \(PE_s\equiv\frac{1}{2}k\Delta\vec{x}^2\) & Yields elastic potential energy \\
  \(\left(KE + PE_g + PE_s\right)_i = \left(KE + PE_g + PE_s\right)_f\) & The Law of Conservation of Energy for springs \\
  \(W_{nc} = \left(KE + PE_g + PE_s\right)_f - \left(KE + PE_g PE_s\right)_i\) & Yields the change in mechanical energy when nonconservative forces are present \\
  \(E = \frac{1}{2}kA^2 = \frac{1}{2}m\vec{v}^2 + \frac{1}{2}k\Delta\vec{x}^2\) & Yields the total mechanical energy $E$ of an object undergoing periodic motion \\

 \tablesubsection{Period \& Frequency}

 \(\vec{v}_0 = \displaystyle\frac{2\pi A}{T}\) & Yields the constant velocity $\vec{v}_0$ of an object around a circular path where $T$ is the period \\
 \(T = 2\pi\displaystyle\sqrt{\frac{m}{k}}\) & Yields the period $T$ in \si{\second} of an object in SHM on a spring \\
 \(f = \displaystyle\frac{1}{T}\) & Yields the frequency $f$ in \si{\hertz} Hertz of an object in SHM \\
 \(f = \displaystyle\frac{1}{2\pi}\sqrt{\frac{k}{m}}\) & Yields the frequency of an object in SHM on a spring \\
 \(\vec{\omega} = 2\pi f = \displaystyle\sqrt{\frac{k}{m}}\) & Yields the angular frequency $\vec{\omega}$ of an object in SHM \\

 \tablesubsection{Position, Velocity, \& Acceleration as a Function of Time}

 \(x = A\cos\left(2\pi ft\right)\) & Yields the $x$-position of an object moving in SHM \\
 \(\vec{v} = -A\omega\sin\left(2\pi ft\right)\) & Yields the velocity of an object moving in SHM \\
 \(\vec{a} = -A\omega^2\cos\left(2\pi ft\right)\) & Yields the acceleration of an object moving in SHM \\

 \tablesubsection{Motion of a Pendulum}

 \(\vec{F}F_t = -mg\sin\theta = -mg\sin\displaystyle\left(\frac{s}{L}\right)\) & Yields the force acting tangent to the circular arc of the pendulum where $s$ is the displacement of the pendulum from equilibrium and $L$ is the length of the pendulum \\
 \(T = 2\pi\displaystyle\sqrt{\frac{\vec{L}}{g}}\) & Yields the period of a pendulum \\
 \(T = 2\pi\displaystyle\sqrt{\frac{I}{mg\vec{L}}} = 2\pi\sqrt{\frac{\vec{L}}{g}}\) & Yields the period of a \textit{physical pendulum}, a pendulum of an object of any shape (e.g., a potato) which pivots about point $O$ which is a distance $L$ from the object's center of mass where $I=ml^2$ \\

 \tablesubsection{Waves}

 \(\vec{v}=f\lambda=\displaystyle\frac{\lambda}{T}\) & Yields the velocity of a wave where $\lambda$ is the wavelength \\
 \(\vec{v} = \displaystyle\sqrt{\frac{\vec{F}_t}{\mu}}\) & Yields the velocity of a wave moving along a string where $\vec{F}_t$ is the force of tension of the string and $\mu$ is the mass per unit length of the string \\
 
\end{longtable}
%%% Local Variables:
%%% mode: latex
%%% TeX-master: "../main"
%%% End: